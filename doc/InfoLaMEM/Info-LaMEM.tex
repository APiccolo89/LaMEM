% main packages
\RequirePackage{lineno} 
\documentclass[a4paper,11pt]{article}
\usepackage[margin=2cm]{geometry}
\usepackage{amsmath}
\usepackage{natbib}
\usepackage{graphicx}
\usepackage{setspace}
\usepackage{multirow}
\usepackage{datetime}
\usepackage{longtable}

% color
\usepackage{color, colortbl}
\definecolor{Gray}{gray}{0.9}

% hyperref
\usepackage{hyperref}
\hypersetup{
    colorlinks=true, %set true if you want colored links
    linktoc=all,         %set to all if you want both sections and subsections linked
    linkcolor=blue,  %choose some color if you want links to stand out
}

% shortkeys
\newcommand{\degreec}{\ensuremath{^\circ}}
\newcommand{\Pas}{\text{Pa.s}}
\newcommand{\reffig}{Fig. \ref}
\newcommand{\reftab}{Table \ref}
\newcommand{\refeq}{Equation (\ref)}

% other packages
\usepackage{courier}
\usepackage{tabularx}
\usepackage{wrapfig,lipsum,booktabs}
\usepackage{arydshln}
\usepackage{mathabx}
\usepackage{subfigure}
\usepackage{fancyheadings}
\usepackage[titletoc]{appendix}
\bibliographystyle{plainnat}
\usepackage{cleveref}
\usepackage{float}

\def\imagetop#1{\vtop{\null\hbox{#1}}}

% graphics path
\DeclareGraphicsExtensions{.pdf,.png,.jpg}
\graphicspath{{figures/}}

% italics citations
\usepackage{etoolbox}
\makeatletter
\patchcmd{\NAT@test}{\else\NAT@nm}{\else\NAT@nmfmt{\NAT@nm}}{}{}
\let\NAT@up\itshape
\makeatother

% title
\title{User Info for LaMEM canonical\\ \large - please update this document periodically -}
\date{\today}
\author{put together by Adina E. Pusok}

% ---------------------------------- BEGIN DOCUMENT  ---------------------------------- %
\begin{document}
\maketitle
\tableofcontents

\newpage

\pagestyle{fancyplain}
\rhead[\fancyplain{}{\slshape \rightmark}]{\thepage}
\lhead[\thepage]{\fancyplain{}%
{\slshape \leftmark}}
\cfoot{}

%\doublespacing

% ---------------------------------- INSTALLATION  ---------------------------------- %
%%% Installation
\section{Installation and compilation}
Before you proceed with installation or re-installation \textbf{remove/comment ALL PATH} modifications related to \texttt{valgrind/mpi/petsc} in \texttt{.bash\_profile}.
Also comment \texttt{PETSC\_DIR}, \texttt{PETSC\_ARCH} environmental variables.\\

\textbf{Important:} Current version of LaMEM is compiled with \texttt{petsc3.5}.\\

All installation instructions can be found in \texttt{LaMEM/doc/installation/}. \\

% ---------------------------------- INSTALLATION ON MAC  ---------------------------------- %
\subsection{Installation on Mac}

% ---------------------------------- GCC INST.  ---------------------------------- %
\subsubsection{gcc compilers}

\begin{enumerate}
\item Install \texttt{macports} from:

\indent \texttt{http://www.macports.org/install.php}\\

Use Mac OS X Package (.pkg) Installer:\\
\indent \texttt{sudo port -v selfupdate}
   
\item Install compilers and debugger

\indent \texttt{sudo port install gcc47}\\
\indent \texttt{sudo port install gdb}

\item Create links
   
\indent \texttt{sudo ln -s /opt/local/bin/gcc-mp-4.7 /opt/local/bin/gcc}\\
\indent \texttt{sudo ln -s /opt/local/bin/gfortran-mp-4.7 /opt/local/bin/gfortran}\\
\indent \texttt{sudo ln -s /opt/local/bin/g++-mp-4.7 /opt/local/bin/g++} \\ 
\indent \texttt{sudo ln -s /opt/local/bin/ggdb /opt/local/bin/gdb}\\
   
Your compiler commands will be: 

\indent \texttt{gcc}      - C compiler\\
\indent \texttt{g++}      - C++ compiler\\
\indent \texttt{gfortran} - Fortran compiler\\ 
\indent \texttt{gdb}      - GNU debugger

\end{enumerate}

% ---------------------------------- MPICH2 INST.  ---------------------------------- %
\subsubsection{mpich2}

\begin{enumerate}
\item Download:

\indent \texttt{\scriptsize curl http://www.mpich.org/static/tarballs/1.4.1p1/mpich2-1.4.1p1.tar.gz -o mpich2-1.4.1p1.tar.gz}

\item Unpack:

\indent \texttt{tar xvzf mpich2-1.4.1p1.tar.gz}\\
\indent \texttt{cd mpich2-1.4.1p1}

\item Configure and install

\indent \texttt{./configure \textbackslash}\\
\indent \texttt{CC=gcc \textbackslash}\\
\indent \texttt{CXX=g++ \textbackslash}\\
\indent \texttt{F77=gfortran \textbackslash}\\
\indent \texttt{FC=gfortran \textbackslash}\\
\indent \texttt{--prefix=/opt/mpich2 \textbackslash}\\
\indent \texttt{--enable-f77 \textbackslash}\\
\indent \texttt{--enable-fc \textbackslash}\\
\indent \texttt{--enable-cxx \textbackslash}\\
\indent \texttt{--enable-threads=runtime \textbackslash}\\
\indent \texttt{--enable-g=none \textbackslash}\\
\indent \texttt{--enable-fast=O2 \textbackslash}\\
\indent \texttt{--with-thread-package=pthreads \textbackslash}\\
\indent \texttt{--with-pm=hydra}\\

\indent \texttt{make}\\
\indent \texttt{sudo make install}\\

\item Set path in \texttt{.bash\_profile}

\indent \texttt{export PATH=/opt/mpich2/bin:\$PATH}

\end{enumerate}

% ---------------------------------- PETSC INST.  ---------------------------------- %
\subsubsection{petsc}

Current version of LaMEM is compiled with \texttt{petsc3.5}.\\

\begin{enumerate}
\item Download:

\indent \texttt{\scriptsize curl http://ftp.mcs.anl.gov/pub/petsc/release-snapshots/petsc-3.5.0.tar.gz -o petsc-3.5.0.tar.gz}

\item Unpack:

\indent \texttt{tar xvzf petsc-3.5.0.tar.gz}\\
\indent \texttt{cd petsc-3.5.0}

\begin{enumerate}
\item Debug version

\indent \texttt{./configure \textbackslash}\\
\indent \texttt{--prefix=/opt/petsc/petsc-3.5.0-int32-deb \textbackslash}\\
\indent \texttt{--download-fblaslapack=1 \textbackslash}\\
\indent \texttt{--with-debugging=1 \textbackslash}\\
\indent \texttt{--COPTFLAGS="-g -O0" \textbackslash}\\
\indent \texttt{--FOPTFLAGS="-g -O0" \textbackslash}\\
\indent \texttt{--CXXOPTFLAGS="-g -O0" \textbackslash}\\
\indent \texttt{--with-large-file-io=1 \textbackslash}\\
\indent \texttt{--with-cc=mpicc \textbackslash}\\
\indent \texttt{--with-cxx=mpicxx \textbackslash}\\
\indent \texttt{--with-fc=mpif90 \textbackslash}\\
\indent \texttt{--with-shared-libraries=0 \textbackslash}\\
\indent \texttt{--download-metis=1 \textbackslash}\\
\indent \texttt{--download-parmetis=1 \textbackslash}\\
\indent \texttt{--download-ml=1 \textbackslash}\\
\indent \texttt{--download-hypre=1 \textbackslash}\\
\indent \texttt{--download-scalapack=1 \textbackslash}\\
\indent \texttt{--download-mumps=1 \textbackslash}\\
\indent \texttt{--download-superlu\_dist=1 \textbackslash}\\
\indent \texttt{--download-suitesparse=1 }\\


\indent \texttt{make all}

\indent \texttt{sudo make install}

\item Optimized version

\indent \texttt{./configure \textbackslash}\\
\indent \texttt{--prefix=/opt/petsc/petsc-3.5.0-int32-opt \textbackslash}\\
\indent \texttt{--with-blas-lapack-lib="-Wl,-framework,vecLib" \textbackslash}\\
\indent \texttt{--with-debugging=0 \textbackslash}\\
\indent \texttt{--COPTFLAGS="-O3" \textbackslash}\\
\indent \texttt{--FOPTFLAGS="-O3" \textbackslash}\\
\indent \texttt{--CXXOPTFLAGS="-O3" \textbackslash}\\
\indent \texttt{--with-large-file-io=1 \textbackslash}\\
\indent \texttt{--with-cc=mpicc \textbackslash}\\
\indent \texttt{--with-cxx=mpicxx \textbackslash}\\
\indent \texttt{--with-fc=mpif90 \textbackslash}\\
\indent \texttt{--with-shared-libraries=0 \textbackslash}\\
\indent \texttt{--download-metis=1 \textbackslash}\\
\indent \texttt{--download-parmetis=1 \textbackslash}\\
\indent \texttt{--download-ml=1 \textbackslash}\\
\indent \texttt{--download-hypre=1 \textbackslash}\\
\indent \texttt{--download-scalapack=1 \textbackslash}\\
\indent \texttt{--download-mumps=1 \textbackslash}\\
\indent \texttt{--download-superlu\_dist=1 \textbackslash}\\
\indent \texttt{--download-suitesparse=1 }\\

\indent \texttt{make all}

\indent \texttt{sudo make install}

\end{enumerate}

\item Set paths in \texttt{.bash\_profile}:

\indent \texttt{export PETSC\_DEB=/opt/petsc/petsc-3.5.0-int32-deb}\\
\indent \texttt{export PETSC\_OPT=/opt/petsc/petsc-3.5.0-int32-opt}\\

\textbf{Note:} No more \texttt{PETSC\_DIR} and no more switching between the debug and optimized versions. Makefile is now supporting building both versions simultaneously.

\end{enumerate}

% ---------------------------------- INSTALLATION ON CLUSTERS  ---------------------------------- %
\subsection{Installation on clusters}

If modules and/or paths to compilers and petsc are not provided, please refer to installation instructions found in \texttt{LaMEM/doc/installation/}.\\
\\
Currently, just add these modules or paths to .bashrc (Gaia, Sith and Mogon) and logout/login for them to take effect.\\
\\
\textbf{Gaia and Sith:}

Add this to .bashrc:\\
\indent \texttt{\# PETSC 3.5.0}\\
\indent \texttt{export PETSC\_DEB=/opt/petsc/petsc-3.5.0-int32-deb}\\
\indent \texttt{export PETSC\_OPT=/opt/petsc/petsc-3.5.0-int32-opt}\\
\\
\textbf{Mogon:}

Add this to .bashrc:\\
\indent \texttt{\# MODULES}\\
\indent \texttt{module load gcc/4.8.2}\\ 
\indent \texttt{module load mpi/platform\_mpi\_ce/9.1.2}\\ 
\indent \texttt{module load acml/5.3.0/gfortran/gfortran64\_fma4 }\\
\indent \texttt{module load software/petsc/3.5.1/gcc\_4.8.2\_platformmpi\_9.1.2}

% ---------------------------------- LAMEM REPOSITORY  ---------------------------------- %
\subsection{Get a local copy of LaMEM}

To get access to LaMEM repository, you should have an account on Gaia (if it does not work, contact Boris for account info):\\

\indent \texttt{ssh USERNAME@gaia.geo.uni-mainz.de}

\begin{enumerate}
\item If you want to checkout a new LaMEM repository, use this command:

\indent \texttt{\footnotesize svn co svn+ssh://USERNAME@gaia.geo.uni-mainz.de/local/home/lkausb/svn/LaMEM/trunk ./LaMEM}

\item If you have a version of LaMEM that you checked out previously already and are using, you don't have to delete that. Instead, go to the \texttt{./LaMEM} directory and type

\indent \texttt{\footnotesize svn relocate svn+ssh://USERNAME@gaia.geo.uni-mainz.de/local/home/lkausb/svn/LaMEM/trunk}\\

If SVN complains that it doesn't know a 'relocate command', you have an older version of svn and we have to do two steps:

\begin{enumerate}
\item \texttt{\tiny svn switch --relocate svn+ssh://USERNAME@musashi.ethz.ch/var/svn svn+ssh://USERNAME@gaia.geo.uni-mainz.de/local/home/lkausb/svn}
\item \texttt{\scriptsize svn switch svn+ssh://lkausb@gaia.geo.uni-mainz.de/local/home/lkausb/svn/LaMEM/trunk}
\end{enumerate}

\item If you are using  the 'Versions' software, make a right-click on the 'bookmark' of your LaMEM directory, and select "Edit Bookmark". Then, change 'Location' to:\\
\indent \texttt{\footnotesize svn+ssh://USERNAME@gaia.geo.uni-mainz.de/local/home/lkausb/svn/LaMEM/trunk}\\
and change your username and password to the ones on GAIA. After 'save', it should automatically relink it.\\

You can verify that everything is correct by typing:
\indent \texttt{svn info} in the LaMEM directory. It should have as URL:\\
\indent \texttt{\footnotesize URL: svn+ssh://username@gaia.geo.uni-mainz.de/local/home/lkausb/svn/LaMEM/trunk}

\end{enumerate}

% ---------------------------------- COMPILING LAMEM  ---------------------------------- %
\subsection{Compilation of LaMEM}

Once the compilers and pets have been installed, you can now compile LaMEM. Go to \texttt{./LaMEM/src} directory and do the following:\\

\textbf{debug mode}: \indent \texttt{make mode=deb all}\\ 
or simply \texttt{make all}, \texttt{mode=deb} is the default to build everything in debug mode.\\

\textbf{optimized mode}: \indent \texttt{make mode=opt all}\\
to build everything in optimized mode.\\

\textbf{Note:} No more \texttt{PETSC\_DIR} and no more switching between the debug and optimized versions with \texttt{.bash\_profile}. Makefile is now supporting building both versions simultaneously.

\begin{enumerate}
\item SVN\\
Also it's not necessary to do svn updates explicitly.\\
\indent \texttt{make update} 
to update to latest version, and to write a new Version.h file.

\item VIEW ENVIRONMENT\\
\indent \texttt{make mode=deb print}\\
or \\
\indent \texttt{make mode=opt print} \\
to view your environmental variables.

\item TESTS (not yet in LaMEM canonical)\\
\\
Running tests, type in the \texttt{./LaMEM/tests} directory:\\
\indent \texttt{make mode=deb check} to perform tests in debug mode.\\
\indent \texttt{make mode=opt check} or (or simply \texttt{make check}, \texttt{mode=opt} is the default) to perform tests in optimized mode.

\textbf{Note:} Take care that \texttt{mode=deb} is the default for LaMEM, and \texttt{mode=opt} is the default for tests.
\end{enumerate}

% ---------------------------------- RUNNING LAMEM ---------------------------------- %
\subsection{Running LaMEM}

The LaMEM executable is located in \texttt{/LaMEM/bin/deb/} for the debug version and \texttt{/LaMEM/bin/opt/} for the optimized version. You can run LaMEM by typing (give the path to the right LaMEM executable):\\
\indent \texttt{mpiexec -n <no\_cpu> LaMEM -ParamFile <input\_file> <command\_line\_options>}\\

% ---------------------------------- PHYSICS ---------------------------------- %
%%% PHYSICS
\section{Physics}

%%% SOLVERS
%\section{Solvers}

% ---------------------------------- INPUT PARAMETERS ---------------------------------- %
\newpage
%%% INPUT
\section{Input Parameters}

Examples of input files can be found in \texttt{/LaMEM/input\_models/PROJECTS\_FDSTAG\_CANONICAL/}. In this sections, all possible input parameters are explained.

% ---------------------------------- GENERAL INPUT PARAMETERS  ---------------------------------- %
\subsection{General input parameters}
\indent \textit{Setting Input Parameters in input file and/or from command line}\\

The general command to run LaMEM is:\\
\indent \texttt{mpiexec -n <no\_cpu> LaMEM -ParamFile <input\_file> <extra command line options>}\\

\textbf{NOTE:}\\
\indent In general, parameters can be specified in the input file or from the command line. The command line options need to have a "\texttt{-}" prefix (i.e. \texttt{-L 10 -W 100 -H 90}). 

\indent If multiple definitions of the same parameter, command line options overwrite input file options! Some input parameters are not needed anymore, such as: \texttt{num\_phases}. \\

\indent Table \ref{tab:input0} lists all major parameters. Please add new LaMEM parameters or options to this table.

%%% Input table
\begin{table}[!htbp]
\begin{center}
\scriptsize
{\renewcommand{\arraystretch}{1.3}
  \begin{longtable}{ p{4cm} l c p{7cm} }
  \hline
Variable &Name&Type&Comments \\
\hline
\texttt{nel\_x}&\# of cells in x-dir&int&Negative number implies corresponding number of mesh segments for variable grid spacing\\
\texttt{nel\_y}&\# of cells in y-dir&int& same as for \texttt{nel\_x} \\
\texttt{nel\_z}&\# of cells in z-dir&int& same as for \texttt{nel\_x} \\
\texttt{seg\_x}&&scalar array& Array includes: 1. coordinates of the delimiters between the segments (n-1 points); 2. number of cells (n points); 3. bias coefficients (n points). See \hyperref[sec:variablegridspacing]{Variable grid spacing}\\
\texttt{seg\_y}&&scalar array& same as for \texttt{seg\_x}\\
\texttt{seg\_z}&&scalar array& same as for \texttt{seg\_x}\\
\texttt{DimensionalUnits}&dimensional indicator&int&0 - non-dimensional, 1 - dimensional; See \hyperref[sec:scaling]{Scaling} \\
\texttt{L}&length (y-dir)&scalar& domain size in y-dir\\
\texttt{W}&width (x-dir)&scalar& domain size in x-dir\\
\texttt{H}&height (z-dir)&scalar& domain size in z-dir\\
\texttt{x\_left}&coord of left corner&scalar&the x-coords: \texttt{(x\_left:nel\_x:x\_left+W)} \\
\texttt{y\_front}&coord of front corner&scalar& the y-coords: \texttt{(y\_front:nel\_y:y\_front+L)}\\
\texttt{z\_bot}&coord of bottom corner&scalar& the z-coords: \texttt{(z\_bottom:nel\_z:z\_bottom+H)}\\
\texttt{msetup}&model setup&string&See \hyperref[sec:modelsetup]{Model Setup}\\
\texttt{ParticleFilename}&name of markers file&string&See (parallel and redundant input) \\
\texttt{\tiny LoadInitialParticlesDirectory}&name of directory&string&directory where the marker files are \\
\texttt{NumPartX}&\# of markers/cell (x-dir)&int& min = 2 \\
\texttt{NumPartY}&\# of markers/cell (y-dir)&int& min = 2 \\
\texttt{NumPartZ}&\# of markers/cell (z-dir)&int& min = 2 \\
\texttt{OutputFile}&name of output file&string& \\
\texttt{save\_timesteps}&save every \# timesteps&int& example: \texttt{save\_timesteps = 10} will save output every 10 time steps \\
\texttt{time\_end}&\# of timesteps&int&the simulation will finish after (0+\texttt{time\_end}) \\
\texttt{save\_breakpoints}&save every \# breakpoints&int&needed to restart simulation; example: \texttt{save\_breakpoints = 10} will save breakpoint files every 10 time steps \\
\texttt{SaveParticles}&save markers indicator&int&1 - save markers to file after every time step \\
\texttt{-AddRandomNoiseParticles}&flag for random noise&int&if \texttt{TRUE}, will add some random noise to marker distribution \\
\texttt{-restart}&restart flag&int&0-no restart; 1-restart simulation from the last saved breakpoint. If \texttt{restart = 1}, but no breakpoint files were found, it will start the simulation from beginning. \\
\texttt{-SavePartitioning}&proc partitioning flag&bool&if \texttt{TRUE}, the domain partitioning between processors will be printed to file. Warning: simulation will be stopped after the file was created. \\
\texttt{CFL}&Courant criterion&scalar&should be < 1 (generally taken as 0.5) \\
\texttt{dt\_max}&max. time step&scalar& time step is calculated internally with the CFL, but this is the maximum admisable\\
\texttt{DII\_ref}&initial reference strain rate&scalar&needed for initial guess in non-linear solve \\
\texttt{LowerViscosityCutoff}&lower viscosity cutoff&scalar&min viscosity allowed in the simulation \\
\texttt{UpperViscosityCutoff}&upper viscosity cutoff&scalar&max viscosity allowed in the simulation \\
\texttt{InitViscosity}&initial reference viscosity&scalar&needed for initial guess in non-linear solve  \\
\texttt{<Characteristic Values>}&Scaling&struct&See \hyperref[sec:scaling]{Scaling} \\
\texttt{<BoundaryConditions>}&Boundary conditions&struct&See \hyperref[sec:bc]{Boundary Conditions} \\
\texttt{<Output>}&Output&struct&See \hyperref[sec:output]{Output} \\
\texttt{<Pushing>}&Pushing BC&struct&See \hyperref[sec:pushing]{Pushing BC} \\
\texttt{<Softening Laws>}&Softening Laws&struct&See \hyperref[sec:softlaws]{Softening Laws} \\
\texttt{<Material Properties>}&Material parameters&struct&See \hyperref[sec:material]{Material Properties} \\
\texttt{<PetscOptions>}&Petsc and solver options&struct&See \hyperref[sec:petsc]{Petsc and Solvers} \\
\texttt{\tiny -use\_quasi\_harmonic\_viscosity}&flag &bool& flag to compute quasi-harmonic viscosities\\
%%\texttt{}&&& \\
\hline  

  \end{longtable}
  }
  \addtocounter{table}{-1}
  \caption{Major parameters to be set in input file/command line. Note, if you set them from the command line, every parameter name should have a "\texttt{-}" prefix (i.e. \texttt{-L 10 -W 100 -H 90})}
  \label{tab:input0}
  \end{center}
\end{table}

% ----------------------------------VARIABLE GRID SPACING  ---------------------------------- %
\subsection{Variable grid spacing}
\label{sec:variablegridspacing}

% ---------------------------------- PETSC AND SOLVER OPTIONS ---------------------------------- %
\subsection{PETSc and solver options}
\label{sec:petsc}

%%%% Solvers table
%\begin{table}[!h]
%\begin{center}
%\footnotesize
%{\renewcommand{\arraystretch}{1.3}
%  \begin{tabular}{ l l c l }
%  \hline
%Variable (1) &Name&Type&Comments \\
%\hline
%\texttt{<PetscOptions>}&&& \\
%\texttt{-res\_log}&&& \\
%\texttt{-darcy\_test}&&& \\
%\texttt{-load\_init}&&& \\
%\texttt{-test\_coupled}&&& \\
%\texttt{-test\_uncoupled}&&& \\
%\texttt{-dump\_init}&&& \\
%\texttt{-pcmat\_no\_dev\_proj}&&& \\
%\texttt{-gmg\_pc\_view}&&& \\
%\texttt{-gmg\_dump}&&& \\
%\texttt{-options\_left}&&& \\
%%%\texttt{}&&& \\
%\hline  
%
%  \end{tabular}
%  }
%  \caption{Options and variables to be set in input file/command line.}
%  \label{tab:solver}
%  \end{center}
%\end{table}

% ---------------------------------- BOUNDARY CONDITIONS  ---------------------------------- %
\subsection{Boundary Conditions}
\label{sec:bc}

%\begin{table}[!h]
%\begin{center}
%\footnotesize
%{\renewcommand{\arraystretch}{1.3}
%  \begin{tabular}{ l l c l }
%  \hline
%Variable (1) &Name&Type&Comments \\
%\hline
%\texttt{BC.Exx}&&& \\
%\texttt{BC.Eyy}&&& \\
%\texttt{BC.LeftBound}&&& \\
%\texttt{BC.RightBound}&&& \\
%\texttt{BC.FrontBound}&&& \\
%\texttt{BC.BackBound}&&& \\
%\texttt{BC.LowerBound}&&& \\
%\texttt{BC.UpperBound}&&& \\
%\texttt{Temp\_top}&&& \\
%\texttt{Temp\_bottom}&&& \\
%\hline  
%
%  \end{tabular}
%  }
%  \caption{Options and variables to be set in input file/command line.}
%  \label{tab:input2}
%  \end{center}
%\end{table}

% ---------------------------------- PUSHING BC  ---------------------------------- %
\subsection{Pushing BC}
\label{sec:pushing}

%\begin{table}[!h]
%\begin{center}
%\footnotesize
%{\renewcommand{\arraystretch}{1.3}
%  \begin{tabular}{ l l c l }
%  \hline
%Variable (2) &Name&Type&Comments \\
%\hline
%\texttt{AddPushing}&&& \\
%\texttt{Pushing.num\_changes}&&& \\
%\texttt{Pushing.time}&&& \\
%\texttt{Pushing.V\_push}&&& \\
%\texttt{Pushing.dir}&&& \\
%\texttt{Pushing.omega}&&& \\
%\texttt{Pushing.coord\_advect}&&& \\
%\texttt{Pushing.reset\_pushing\_coord}&&& \\
%\texttt{Pushing.theta}&&& \\
%\texttt{Pushing.L\_block}&&& \\
%\texttt{Pushing.W\_block}&&& \\
%\texttt{Pushing.H\_block}&&& \\
%\texttt{Pushing.x\_center\_block}&&& \\
%\texttt{Pushing.y\_center\_block}&&& \\
%\texttt{Pushing.z\_center\_block}&&& \\
%\hline  
%  \end{tabular}
%  }
%  \caption{Options and variables to be set in input file/command line.}
%  \label{tab:input3}
%  \end{center}
%\end{table}

% ---------------------------------- OUTPUT  ---------------------------------- %
\subsection{Output}
\label{sec:output}

% ---------------------------------- SCALING  ---------------------------------- %
\newpage
% Scaling
\subsection{Scaling, characteristic values and units}
\label{sec:scaling}

The units in the input file should be consistent with the type of units requested. \\

To run a \textbf{non-dimensional setup} set these parameters in the input file:\\
\begin{tabular}{l l}
\indent \texttt{DimensionalUnits} &\texttt{= 0} \\
\indent \texttt{units} &\texttt{= none} \\
\end{tabular}
\\

To run a \textbf{dimensional setup}, set:\\
\begin{tabular}{l l}
\indent \texttt{DimensionalUnits} &\texttt{= 1} \\
\indent \texttt{units} &\texttt{= si  \# all input parameters are given in SI units}\\
&\\
\indent OR&\\
&\\
\indent \texttt{DimensionalUnits} &\texttt{= 1} \\
\indent \texttt{units} &\texttt{= geo  \# all input parameters are given in geological units}\\ 
\end{tabular}\\
\\
If \texttt{DimensionalUnits = 1}, the following parameters (characteristic values) \textbf{must} also be specified in input:

\begin{tabular}{l l}
\indent \texttt{Characteristic.Length} &\texttt{= 1 \# depends on domain size } \\
\indent \texttt{Characteristic.Viscosity} &\texttt{= 1  \# }\\
\indent \texttt{Characteristic.Temperature} &\texttt{= 1 \# } \\
\indent \texttt{Characteristic.Stress} &\texttt{= 1  \#  }\\ 
\end{tabular}\\
\\
%%% Scaling table 2
\begin{wraptable}[19]{r}{8cm}
%\begin{center}
\footnotesize
{\renewcommand{\arraystretch}{1.3}
  \begin{tabular}{| c | c c c |}
  \hline
  &\multicolumn{3}{c |}{ \texttt{units}} \\
  \hline  
Parameter&\texttt{none}&\texttt{si}&\texttt{geo} \\
\hline
mass&-&kg&kg\\
time&-&s&Myr\\
length&-&m&km\\
temperature&-&K&C\\
force&-&N&N\\
angle&-&deg&deg\\
velocity&-&m/s&cm/yr\\
acceleration&-&m/s$^2$&m/s$^2$\\
stress&-&Pa&MPa\\
strain rate&-&1/s&1/s\\
heat flux&-&W/m$^2$&mW/m$^2$\\
dissipation rate&-&W/m$^3$&W/m$^3$\\
angular velocity&-&deg/s&deg/Myr\\
density&-&kg/m$^3$&kg/m$^3$\\
viscosity&-&Pa.s&Pa.s\\
\hline
&\multicolumn{3}{l |}{Derived units}\\
\hline
gravity strength&\multicolumn{3}{l |}{= force/mass}\\
energy&\multicolumn{3}{l |}{= force*length}\\
power&\multicolumn{3}{l |}{= energy/time}\\
specific heat&\multicolumn{3}{l |}{= energy/mass/temperature}\\
conductivity&\multicolumn{3}{l |}{= power/length/temperature}\\
heat production&\multicolumn{3}{l |}{= power/mass}\\
expansivity&\multicolumn{3}{l |}{= 1/temperature}\\
pressure sensitivity&\multicolumn{3}{l |}{= temperature/stress}\\
  \hline  
  \end{tabular}
  }
  \caption{Setting dimensional units in input file.}
  \label{tab:scaling2}
  %\end{center}
\end{wraptable}

Characteristic values are dependent on model setup (check how they are set in the examples in \texttt{/LaMEM/input\_models/}). They are needed to perform the non-dimensionalization (scaling) of the model. For this, all input parameters, characteristic values and material parameters should have consistent units!\\

\reftab{tab:scaling2} shows in which units parameters should be given. \\
\\
Note: \\
1. Degrees for angles are the preferred units, also for \texttt{units = si}. \\
2. Acceleration should be given in units of m/s$^2$ when dimensional (i.e. g = 10 m/s$^2$).\\
3. Markers from file (redundant, parallel) should have consistent units for coordinates and temperature (see \reftab{tab:scaling2} ).

% ---------------------------------- SOFTENING LAWS  ---------------------------------- %
\newpage
%%% Strain weaking laws 
\subsection{Softening laws}
\label{sec:softlaws}

Every softening law should be defined in the input file in a structure beginning with \texttt{<SofteningStart>} and ending with \texttt{<SofteningEnd>}, and should have a unique phase identifier (\texttt{softID}), a softening ratio (\texttt{A}), a start of softening (\texttt{APS1}) and an end of softening (\texttt{APS2}) such as below. \\
\\
Each parameter is explained in \reftab{tab:soft}. Maximum number of softening laws: 10 (this number can be changes in LaMEM.h - max\_num\_soft).\\

\indent \texttt{<SofteningStart>}\\
\indent \indent \texttt{softID = 1}\\
\indent \indent \texttt{A \hspace{2.7pc}= 0.5}\\
\indent \indent \texttt{APS1 \hspace{0.5pc} = 1}\\
\indent \indent \texttt{APS2 \hspace{0.5pc} = 3}\\
\indent \texttt{<SofteningEnd>}\\

%%% Softening laws table
\begin{table}[!h]
\begin{center}
\footnotesize
{\renewcommand{\arraystretch}{1.3}
  \begin{tabular}{p{3cm} p{3cm} c p{5cm}}
 \hline
\text{Parameters to define}&Definition&Dimen. values&Comments \\
 \hline  
\texttt{softID}&soft law id&0 - 9&always set this parameter\\
\texttt{A}&softening ratio&[0,1]&\texttt{A = 1 - a/a0}, where \texttt{a} - is cohesion(or friction angle) after softening, and \texttt{a0} is cohesion (or friction angle) before softening and specific for each material\\
\texttt{APS1}&weakening plastic strain (begin)&&\\
\texttt{APS2}&weakening plastic strain (end)&&\\
 \hline
     \end{tabular}
     }
  \caption{List of parameters needed to define a softening law.}
  \label{tab:soft}
  \end{center}
\end{table}

% ---------------------------------- MATERIAL PARAMETERS ---------------------------------- %
%%% Material Parameters
\subsection{Material parameters}
\label{sec:material}
\indent \textit{Setting Material Parameters in input file}\\

Every material phase should be defined in the input file in a structure beginning with \texttt{<MaterialStart>} and ending with \texttt{<MaterialEnd>}, and should have a unique phase identifier (\texttt{ID}), such as below. The principle of setting material parameters, is that, if you specify it, it will be active; you don't specify it, it's not going to be active. All the material parameters and cases are listed in \reftab{tab:matprops}. Maximum number of phases: 32 (this number can be changes in LaMEM.h - max\_num\_phases).\\

\indent \texttt{<MaterialStart>}\\
\indent \indent \texttt{ID = 1}\\
\indent \indent \texttt{...}\\
\indent \texttt{<MaterialEnd>}

\subsubsection{Examples}

1) Viscous falling block with 2 material phases (non-dimensional)\\ 
{\small
\indent \texttt{<MaterialStart>}\\
\indent \indent \texttt{ID = 0}\\
\indent \indent \texttt{rho0 = 1}\\
\indent \indent \texttt{eta = 1}\\
\indent \texttt{<MaterialEnd>}\\

\indent \texttt{<MaterialStart>}\\
\indent \indent \texttt{ID = 1}\\
\indent \indent \texttt{rho0 = 2}\\
\indent \indent \texttt{eta = 1e3}\\
\indent \texttt{<MaterialEnd>}\\}

2) Power-law material (non-dimensional)\\ 
{\small
\indent \texttt{<MaterialStart>}\\
\indent \indent \texttt{ID = 0}\\
\indent \indent \texttt{rho0 = 1}\\
\indent \indent \texttt{Bn = 0.5}\\
\indent \indent \texttt{n = 3}\\
\indent \texttt{<MaterialEnd>}\\
\indent \indent OR\\
\indent \texttt{<MaterialStart>}\\
\indent \indent \texttt{ID = 0}\\
\indent \indent \texttt{rho0 = 1}\\
\indent \indent \texttt{eta0 = 1e3}\\
\indent \indent \texttt{e0 = 1e-5}\\
\indent \indent \texttt{n  = 3}\\
\indent \texttt{<MaterialEnd>}\\}

3) Defining a visco-elasto-plastic material phase with T-dependence (no strain weakening or radiogenic heat production)\\

\begin{tabular}{l l l}
\small
\indent \texttt{<MaterialStart>}&&\\
\indent \indent \texttt{ID}&\texttt{= 0}&\texttt{\# phase id}\\
\indent \indent \texttt{rho0 }&\texttt{= 3000}&\texttt{\# density [kg/m3]}\\
&&\\
\indent \indent \texttt{\# creep}&&\\
\indent \indent \texttt{eta  }&\texttt{= 1e20}&\texttt{\# 	ref. viscosity [Pa.s]}\\
\indent \indent \texttt{eta0 }&\texttt{= 1e18}&\texttt{\# 	ref. viscosity [Pa.s]}\\
\indent \indent \texttt{e0 }&\texttt{= 1e-15}&\texttt{\# ref. strain rate [1/s]}\\
\indent \indent \texttt{n }&\texttt{= 3}&\texttt{\# power-law exponent}\\
\indent \indent \texttt{Ed}&\texttt{= 3e5}&\texttt{\# activation energy [J/mol]}\\
\indent \indent \texttt{En}&\texttt{= 5.4e5}&\texttt{\# activation energy [J/mol]}\\
&&\\
\indent \indent \texttt{\# elasticity}&&\\
\indent \indent \texttt{shear }&\texttt{= 5e10}&\texttt{\# elastic shear modulus [Pa]}\\
&&\\
\indent \indent \texttt{\# plasticity}&&\\
\indent \indent \texttt{cohesion }&\texttt{= 4e6}&\texttt{\# cohesion [Pa]}\\
\indent \indent \texttt{friction }&\texttt{= 30}&\texttt{\# friction angle [deg]}\\
&&\\
\indent \indent \texttt{\# temp}&&\\
\indent \indent \texttt{alpha }&\texttt{= 1e-5}&\texttt{\# coeff. of thermal expansion [1/K]}\\
\indent \indent \texttt{cp }&\texttt{= 1.2e3}&\texttt{\# heat capacity [J/kg/K]}\\
\indent \indent \texttt{k }&\texttt{= 2.5}&\texttt{\# thermal conductivity [W/m/K]}\\
\indent \texttt{<MaterialEnd>}\\
\end{tabular}\\

%%% Errors
\subsubsection{Errors related to material properties input}
\begin{enumerate}
\item Phase \texttt{ID} is not specified
\item If \texttt{rho0} is not specified - this is an essential parameter
\item If no creep law is specified (i.e. \texttt{Bd, Bn, and Bp} are all 0). Make sure you set parameters for at least one creep law!
\item Power-law: 
	\begin{enumerate}
	\item If specified (\texttt{eta0, e0}) or (\texttt{Bn}), but not \texttt{n}.
	\item If specified \texttt{n}, but not all/any (\texttt{eta0, e0}) or (\texttt{Bn}).
	\end{enumerate}
\item Peierls creep: if \texttt{Bp} is not defined, but any of the others Peierls parameters is.
\item Plasticity: if \texttt{cohesion} and \texttt{friction} are not defined together.
\item Temperature: if \texttt{cp} and \texttt{k} are not defined together.
\end{enumerate}

% Predefined profiles
\subsubsection{Predefined rheological profiles}
\label{sec:predefined}

Values for each profile can be found in \texttt{matProps.c - SetDiffProfile()}, \texttt{SetDislProfile()} and \texttt{SetPeirProfile()}. \textbf{Please check the values in the respective papers/studies when you are using them!}

\begin{enumerate}

\item Diffusion creep - the parameter \texttt{diff\_profile} can be set in the following way:
\begin{itemize}
\item \texttt{Dry\_Olivine\_diff\_creep-Hirth\_Kohlstedt\_2003}
\item \texttt{Wet\_Olivine\_diff\_creep-Hirth\_Kohlstedt\_2003\_constant\_C\_OH}
\item \texttt{Wet\_Olivine\_diff\_creep-Hirth\_Kohlstedt\_2003}
\end{itemize}

\item Dislocation creep - the parameter \texttt{disl\_profile} can be set in the following way:
\begin{itemize}
\item \texttt{Dry\_Olivine-Ranalli\_1995}
\item \texttt{Wet\_Olivine-Ranalli\_1995}
\item \texttt{Quartz\_Diorite-Hansen\_Carter\_1982}
\item \texttt{Diabase-Caristan\_1982}
\item \texttt{Tumut\_Pond\_Serpentinite-Raleigh\_Paterson\_1965}
\item \texttt{Wet\_Quarzite-Ranalli\_1995}
\item \texttt{Quarzite-Ranalli\_1995}
\item \texttt{Mafic\_Granulite-Ranalli\_1995}
\item \texttt{Maryland\_strong\_diabase-Mackwell\_et\_al\_1998}
\item \texttt{Wet\_Quarzite-Ueda\_et\_al\_2008}
\item \texttt{Diabase-Huismans\_et\_al\_2001}
\item \texttt{Granite-Huismans\_et\_al\_2001}
\item \texttt{Dry\_Upper\_Crust-Schmalholz\_Kaus\_Burg\_2009}
\item \texttt{Weak\_Lower\_Crust-Schmalholz\_Kaus\_Burg\_2009}
\item \texttt{Plagioclase\_An75-Ranalli\_1995}
\item \texttt{Wet\_Olivine\_disl\_creep-Hirth\_Kohlstedt\_2003}
\item \texttt{Wet\_Olivine\_disl\_creep-Hirth\_Kohlstedt\_2003\_constant\_C\_OH}
\item \texttt{Dry\_Olivine\_disl\_creep-Hirth\_Kohlstedt\_2003}
\item \texttt{Olivine-Burg\_Podladchikov\_1999}
\item \texttt{Wet\_Upper\_Mantle-Burg\_Schmalholz\_2008}
\item \texttt{Granite-Tirel\_et\_al\_2008}
%\indent \texttt{}\\
\end{itemize}

\item Peierls creep - the parameter \texttt{peir\_profile} can be set in the following way:
\begin{itemize}
\item \texttt{Olivine\_Peierls-Kameyama\_1999}
\end{itemize}

\end{enumerate}

%%% Material Parameters table
\begin{table}[H]
\begin{center}
\footnotesize
{\renewcommand{\arraystretch}{1.3}
  \begin{tabular}{p{2.7cm} p{2cm} p{3cm} c p{5cm}}
 \hline
Case&\text{Parameters}&Definition&Dimen. values SI [GEO]&Comments \\
 \hline  
%&\textit{ID}&phase id&0 - 30&always set this parameter\\
constant density&\texttt{rho0}&reference density&0 - 5000 kg/m$^3$&\\

\hdashline
linear viscous&\texttt{diff\_profile}&\hyperref[sec:predefined]{set predefined profile}&string&profiles from the literature - \\
(diffusion creep)&\textbf{OR}&&&has priority over other definitions\\
&\texttt{eta}&reference viscosity&1e-10 - 1e28 Pa.s&either (\texttt{eta}) or (\texttt{Bd}) can be defined;\\
&\textbf{OR}&&& if both \texttt{eta} and \texttt{Bd} are defined, \texttt{Bd} has priority\\
&\texttt{Bd}&diff. creep coeff.&1/(Pa.s)& \texttt{Bd = 1/(2*eta)}\\
&\texttt{Ed}&diff. creep activation energy&3.0e5 J/mol&(*T) \\
&\texttt{Vd}&diff. creep activation volume&m$^3$/mol&(*T) \\

\hdashline
power-law &\texttt{disl\_profile}&\hyperref[sec:predefined]{set predefined profile}&string&profiles from the literature - \\
(dislocation creep)&\textbf{OR}&&&has priority over other definitions\\
&\texttt{eta0}&ref. viscosity&1e-10 - 1e28 Pa.s&either define (\texttt{eta0, e0, n}) or  \\
&\texttt{e0}&ref. strain rate&1e-15 s$^{-1}$&(\texttt{Bn, n}); if all are defined, \texttt{Bn} has priority;\\
&\texttt{n}&power-law exponent&1 - 10&\texttt{e0} and \texttt{n} must be defined together\\
&\textbf{OR}&&&\\
&\texttt{Bn}&disl. creep coeff.&1/(Pa$^n$s)&\texttt{Bn = (2*eta0)$^{-n}$ e0$^{(1-n)}$} \\
&\texttt{En}&disl. creep activation energy&5.4e5 J/mol&(*T) \\
&\texttt{Vn}&disl. creep activation volume&m$^3$/mol&(*T) \\

\hdashline
Peierls creep&\texttt{Bp}&pre-exponential constant&$\sim$5e11 s$^{-1}$&\\
&\texttt{Ep}&disl. creep activation energy&5.4e5 J/mol&(*T) \\
&\texttt{Vp}&disl. creep activation volume&m$^3$/mol&(*T) \\
&\texttt{taup}&scaling stress&$\sim$ 1e9-1e10 Pa [MPa]& more info - Kameyama et al. 1999 \\
&\texttt{gamma}&approx. parameter&0.1& \\
&\texttt{q}&stress-dep. param.&2& \\

\hdashline
elasticity&\texttt{shear}&elastic shear modulus&Pa [MPa]&if not specified, no elasticity\\
&\texttt{bulk}&elastic bulk modulus&Pa [MPa]&\\
&\texttt{Kp}&pressure-dep. param&-&\\

\hdashline
plasticity &\texttt{cohesion}&cohesion&10e6 Pa [MPa]&\texttt{cohesion} and \texttt{friction} must be \\
(Drucker-Prager)&\texttt{friction}&friction angle&0$^\circ$ - 45$^\circ$ (30$^\circ$)&defined together; if not specified, no plasticity\\
&\texttt{chSoftID}&ID for softening law&$<$ numSoft&Strain weakening parameter\\
&\texttt{frSoftID}&ID for softening law&$<$ numSoft&Strain weakening parameter\\


\hdashline
temperature  (*T)&\texttt{alpha}&coefficient of thermal expansion&10$^{-5}$K$^{-1}$&\texttt{rho=rho0*(1-alpha(T-T0))}; where \texttt{T0} = [0$^\circ$C, 273.15K]\\
&\texttt{cp}&specific heat (capacity at ct. pressure)&J/kg/K&\\
&\texttt{k}&thermal conductivity&W/m/K&\\
&\texttt{A}&radiogenic heat production&10$^{-8}$-10$^{-8}$ W/m$^3$ &\\

 \hline
     \end{tabular}
     }
  \caption{List of material parameters that can be specified for a material phase. \textbf{Notes: 1.} (*T) Temperature-dependent constitutive laws are active only when 'energy' equation is coupled. \textbf{2.} Activation energy and volume can be specified for every creep law. \textbf{3.} Units are given in SI units. When 'units=geo' use the units given in square brackets when present. }
  \label{tab:matprops}
  \end{center}
\end{table}

% ---------------------------------- MODEL SETUP  ---------------------------------- %
%\newpage
%%% Model Setup
\section{Model Setup}
\label{sec:modelsetup}

% ---------------------------------- VISUALIZATION  ---------------------------------- %
%%% VISUALIZATION
\section{Visualization}

% ---------------------------------- END DOCUMENT  ---------------------------------- %

% references - should have a references.bib file
\bibliography{references}

\end{document}